%=========PROBLEM 4============================================
\section*{Problem 4}
\label{sec:problem4}
All computers treat numbers as approximation to the nearest bit representation of that number. It will greatly differ by the amount of bits used and the type of number. Numbers that are type \textit{int} are always exact but that is not the case for example to numbers of type \textit{float}. This exercise asks to find the minimum value of $\epsilon$ that will satisfy the following conditions and considering data type of float32 and float64:
\begin{enumerate}
    \item[a)] $1 + \epsilon > 1$
    \item[b)] $1 - \epsilon < 1$
    \item[c)] $10^16 +\epsilon > 10^16$
    \item[d)] $10^16 -\epsilon < 10^16$
\end{enumerate}

The steps to solve this problem are the following:

\begin{enumerate}
    \item Initialize my variables.
    \item Evaluate the condition for the first time and save it as prev\_eps.
    \item Evaluate the condition. 
    \label{item:Init}
    If this condition is different than the previous condition than continue to step \ref{item:DiffTrue}, if not continue to step \ref{item:DiffFalse}.
    \item Calculate the new step, h:
    \label{item:DiffTrue}
        \begin{equation}
            h = \frac{\epsilon_0 - \epsilon_1}{2}
        \end{equation}
    \item Calculate new $\epsilon$ with,
        \label{item:DiffFalse}
        \begin{equation}
            \epsilon_1=\epsilon_0 + h
        \end{equation}
        Then continue to step \ref{item:Init}
    \item 
    
Every iteration across my threshold, I would reduce my step by 2.  
    
\end{enumerate}

Table \ref{tab:epsilon} shows the different values of $\epsilon$ that satisfies the condition using either float32 and float64. In the first two cases, the values are different within each other. This could be because 1 only needs one bit for its representation so the decimals and be stored in the rest of the bits. In the second two cases, the values are equals within each same data type. This could be because $10^{16}$ is so big that float32 doesn't get enough bit space left. 

\begin{table}[]
    \centering
    \begin{tabular}{c|l|l}
        CONDITION            & FLOAT 32 BITS            & FLOAT 64 BITS \\
        $1 + \epsilon > 1$  & $5.960465\times 10^{-08}$ & $1.1102230246251568 \times 10^{-16}$\\
        $1 - \epsilon < 1$ & $2.9802322\times 10^{-08}$ & $5.551115123125783 \times 10^{-17}$ \\
        $10^{16} +\epsilon > 10^{16}$ & $536871000.0$ & $1.0000000000000002$ \\
        $10^{16} -\epsilon < 10^{16}$ & $536871000.0$ & $1.0000000000000002$\\
    \end{tabular}
    \caption{Shows the smallest value of $\epsilon$} that satisfies the condition given at different data type.
    \label{tab:epsilon}
\end{table}